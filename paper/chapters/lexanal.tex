%!TEX root = ../main.tex
\section{Lexical analysis}

%
% SUBSECTION
%
\subsection{Line structure}

A SetlX program is built of logical lines.

%#####################
%
\subsubsection{Logical lines}

The end of a logical line is represented by a semicolon `;'.

%#####################
%
\subsubsection{Physical lines}

A physical line consists of a number of characters ending with a end-of-line character. One physical line can have any fraction of a logical line, i.e. one can split code arbitrarily at whitespace.

%#####################
%
\subsubsection{Comments}

Setlx has two types of comments. Comments are ignored by the lexer, they are stripped and do not generate any tokens.

\myparagraph{Singe-line comments}

A single line comment starts with a double slash (\token{//}) that is not part of a string literal, and ends at the end of the physical line. 

\myparagraph{Multi-line comments}

Multi-line comments, start with \token{/*} and end with \token{*/}.

%#####################
%
\subsubsection{Blank lines}

Lines that only contain whitespace are ignored.

%#####################
%
\subsubsection{Whitespace}
Except in string literals, the whitespace characters used to separate tokens are tab, carriage return, form feed, vertical tab.

%
% SUBSECTION
%
\subsection{Identifiers and keywords}

Identifier and functors are unlimited in length. Case is significant.

%#####################
%
\subsubsection{Identifier}

Identifiers are described by the following lexical definitions:

\begin{grammar}
<identifier> ::= <lowercase> <seq> | `_'

<seq> ::= <seq> <char>
\alt $\upepsilon$

<char> ::= <lowercase> | <uppercase> | <digit> | `_'

<lowercase>  ::=  `a'...`z'

<uppercase>  ::=  `A'...`Z'

<digit>      ::=  `0'...`9'
\end{grammar}

%#####################
%
\subsubsection{Functor}

Functors are described by the following lexical definitions:

\begin{grammar}
<functor> ::= <uppercase> <seq>
\end{grammar}

%
% SUBSECTION
%
\subsection{Keywords}
\todo{Update keywords}
The following identifier are treated as keywords in SetlX and cannot used for other purposes:

\begin{table}[h]
\begin{tabular*}{\columnwidth}{@{\extracolsep{\stretch{1}}}*{6}{l}@{}}
\token{true} 		& \token{false} 			& \token{in} 		& \token{notin} 	& \token{forall} 	& \token{exists} 	\\
\token{backtrack} 	& \token{check} 			& \token{match} 	& \token{regex} 	& \token{as} 		& \token{break} 	\\
\token{continue} 	& \token{exit} 				& \token{return} 	& \token{assert} 	& \token{if} 		& \token{else} 		\\
\token{switch} 		& \token{case} 				& \token{default} 	& \token{for} 		& \token{do} 		& \token{while} 	\\
\token{procedure} 	& \token{cachedProcedure} 	& \token{class} 	& \token{static} 	& \token{scan} 		&  \token{using} 	\\
\token{try} 		&  \token{catch} 			& \token{catchUsr} 	& \token{catchLng} 	&  \\
\end{tabular*}
\end{table}

%
% SUBSECTION
%
\subsection{Literals}

Literals are notations for constant values of some built-in types.

%#####################
%
\subsubsection{Character sequences}

\myparagraph{Strings}

Any sequence of characters enclosed in double quotes which does not contain a interpolation is considered a string.

\myparagraph{Literal strings}

Any sequence of characters enclosed in single quotes is considered a literal string.

\myparagraph{String interpolations}

A string interpolation is a string which contains at least one interpolation.

\begin{grammar}
<interpolation> ::= `\$' <expression> `\$'
\end{grammar}

\myparagraph{Escaping}

SetlX supports the same escape sequences as the language C.

%#####################
%
\subsubsection{Numeric literals}

There are two types of numeric literals: integer and floating point numbers. Note that the following definitions do not include signs, that is handled by expressions with unary operators.

\myparagraph{Integer literals}

\begin{grammar}
<integer> ::=  <nonzerodigit> <digits-or-empty> | `0'

<nonzerodigit>   ::=  `1'...`9'

<digits> ::= <digits> <digit>
\alt <digit>

<digits-or-empty> ::= <digits> | $\upepsilon$
\end{grammar}

\myparagraph{Floating point literals}

Floating point literals are described by the following lexical definitions:

\begin{grammar}
<floatnumber>   ::=  <pointfloat> | <exponentfloat>

<pointfloat>    ::=  <fraction>
\alt <intpart> <fraction>
\alt <intpart> `.'

<exponentfloat> ::=  <significand> <exponent>

<significand> ::= <intpart> | <pointfloat>

<intpart>       ::=  <digits>

<fraction>      ::=  `.' <digits>

<exponent>      ::= <e> <sign> <digits>

<e> ::= `e' | `E'

<sign> ::= `+' | `-' | $\upepsilon$

\end{grammar}

%
% SUBSECTION
%
\subsection{Operators}

The following tokens are operators:

\begin{table}[h]
\begin{tabular*}{\columnwidth}{@{\extracolsep{\stretch{1}}}*{6}{l}@{}}
\token{+} 		& \token{-} 	& \token{/} 	& \token{*} 	& \token{\\\\} 	& \token{\%} \\
\token{><} 		& \token{**} 	& \token{\#} 	& \token{@}	& 						 					& \\
\token{<==>} 	& \token{<!=>}	& \token{=>}	& \token{||}	& \token{\&\&} 								& \token{!} \\
\token{==} 		& \token{!=} 	& \token{<} 	& \token{<=} 	& \token{>} 								& \token{>=} \\
\token{\\+} & \token{\\*} & & & &	\\
\end{tabular*}
\end{table}

%
% SUBSECTION
%
\subsection{Delimiters}

The following tokens serve as delimiters in the grammar:

\begin{table}[h]
\begin{tabular*}{\columnwidth}{@{\extracolsep{\stretch{1}}}*{6}{l}@{}}
\token{(} 	& \token{)} 	& \token{[} 	& \token{]} 	& \token{\{}	& \token{\}} 	\\
\token{;} 	& \token{,} 	& \token{:} 	& \token{..}	& \token{.} 	& \token{|} 	\\
\token{:=}	& \token{+=}	& \token{-=} 	& \token{*=}	& 				&				\\
\token{/=}	& \token{\\=} & \token{\%=} &\token{|->}				\\
\end{tabular*}
\end{table}







